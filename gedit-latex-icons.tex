\documentclass[a5j,10pt]{ltjarticle}
\usepackage{fontspec}

% LINK
\usepackage{url}
\usepackage{hyperref}

% 色の使用
\usepackage{xcolor}
\definecolor{code}{HTML}{800000}
\definecolor{mylinkcolor}{RGB}{3, 112, 145} % 色定義
\definecolor{lightblu}{HTML}{004c8c} 
\definecolor{orangish}{wave}{620}
\colorlet{ochre}{blue!30!yellow!70!}

\hypersetup{
    colorlinks=true,
    citecolor=blue,
    linkcolor=lightblu,
    urlcolor=mylinkcolor, % 定義された色 MidnightBlue
}

% 表の使用
\usepackage{float}

% ICONS Lib 読込
\usepackage{lib-gedit-latex-icons} 
\usepackage{graphicx} 
\usepackage{svg} 

\usepackage{datetime2}
\usepackage{setspace}

% FONT-SIZE 定義
\def\fs#1{\fontsize{#1pt}{14pt}\selectfont}

% サンプル定義
% DESKTOP-ICONS
\newcommand{\emblemdebian}{\includegraphics[width=1em]{/usr/share/icons/desktop-base/64x64/emblems/emblem-debian.png}}
% MATERIAL-ICONS
\newcommand{\mdBuild}{\includesvg[width=1em]{./icons/build.svg}}
% BOOTSTRAP-ICONS
\newcommand{\bsgithub}{\includesvg[width=1em]{./icons/github.svg}}

% TITLEPAGE
\title{{\fs{24.88}GEDIT-LaTeX-Plugin \\ ICONS for \LaTeX{}}\\ \vspace{10mm}{\fs{12} \textbf{[ PNG 版 ]}}\vspace{80mm}}
\author{\href{https://github.com/ru-museum/}{ru\_museum \bsgithub}} 
\date{\today}

\begin{document}

\begin{titlepage}
  \maketitle
	\vspace{\fill}
\thispagestyle{empty}
\end{titlepage}

\newpage

\tableofcontents
\clearpage

\newpage

\section{"gedit-latex-plugin"}
\subsection{概要}

\begin{itemize}
  \item LINUX 用エディタ \textbf{gedit} には\LaTeX{}向けのプラグイン \textbf{gedit-latex-plugin} があり、添付の ICON が利用出来ます。\\ 
※ 参照: 「3 gedit-latex ICONS 一覧」
  
  \item ICON 収納 PATH(LINUX Debian の場合):\\
  \textbf{/usr/share/gedit/plugins/latex/icons/} 

\end{itemize}

\subsection{インストールと設定}

\begin{itemize}
  \item インストール\\ 
        \$ apt-get install gedit-latex-plugin (Debian の場合) \\
        {\fs{8} ※ PDF viewer(evince 等)、inkscape(SVG) が必要な場合があります。} 
\vspace{-2mm}
  \item プラグインの設定で SyncTeX をオンにして置きます(作業用)。\\ 
{\fs{8}SyncTeX:gedit と evince(PDF viewer) で LaTeX と PDF を同期します。}
\end{itemize}

\subsection{使用方法}

\subsubsection{ライブラリの読込}
{\textbackslash}usepackage\{lib-gedit-latex-icons\} \% ICON 定義ライブラリ \\ 
 {\textbackslash}usepackage\{graphicx\}\hspace{18mm}\% PNG 用\\ 
 {\textbackslash}usepackage\{svg\}\hspace{26mm}\% SVG 使用時(inkscape 必須)\\ 

\% 色を使用する場合\\
 {\textbackslash}usepackage\{xcolor\}\\
 {\textbackslash}definecolor\{code\}\{HTML\}\{800000\}\\
 {\textbackslash}definecolor\{mylinkcolor\}\{RGB\}\{3, 112, 145\}

\subsubsection{定義}
// フォントサイズのマクロ定義 \\
 {\textbackslash}def{\textbackslash}fs\#1\{{\textbackslash}fontsize\{\#1pt\}\{14pt\}{\textbackslash}selectfont\}\\

// ICON のマクロ定義\\
 {\textbackslash}newcommand\{{\textbackslash}\textbf{gl}closeenv\}\{{\textbackslash}@icon\{{\textbackslash}@dir/close\_env.png\}\}\\
\begin{itemize}\vspace{-6mm}
  \item システムとの衝突を避ける為、prefix として \textbf{gl} を付与しています。
\vspace{-3mm}
 \item command 名はファイル名に準じ \_(アンダーバー)、-(ハイフン) 等は使用出来ませんので削除しています。
\end{itemize}


\subsubsection{表示}

\begin{itemize}

  \item 各々定義した command 名で表示させます。\\
  ※ ファイルは基本 PNG 形式ですので過度の拡大は像が粗くなります。 

\begin{table}[H]
\caption{表示例} %キャプション
\vspace{4mm}
{\hspace{10mm}
\begin{tabular}{lll}
\textbf{サイズ} & \textbf{記述} & \hspace{4mm}\textbf{表示}\\
無指定 & {\textbackslash}glabort & \hspace{8mm}\glabort\\
{\fs{18}18pt} & \{{\textbackslash}fs\{18\} {\textbackslash}glabort\} & \hspace{5mm}{\fs{18} \glabort}\\
{\fs{40}40pt} & \{{\textbackslash}fs\{40\} {\textbackslash}glabort\} & {\fs{40} \glabort}\\
\end{tabular}
}
\end{table}
  
\vspace{-3mm}
  \item \LaTeX{} 付属のシステムアイコンは、\textbf{text-} で使うことが出来ます。\\
{\textbackslash}textparagraph: \textparagraph \\
{\textbackslash}textcopyright : \textcopyright
  \item その他、OS のデスクトップ環境における各種テーマのアイコンセットも同様に使うことが出来ます。\\ 
指定PATH:/usr/share/icons/... (Debian の場合)\\

// 定義例\\
{\textbackslash}newcommand\{{\textbackslash}emblemdebian\}\{/usr/share/icons/desktop-base/\\ 64x64/emblems/emblem-debian.png\}\} \\

\vspace{-6mm}
\begin{table}[H]
\caption{表示例}
\vspace{4mm}
{\hspace{10mm}
\begin{tabular}{lll}
\textbf{サイズ} & \textbf{記述} & \hspace{4mm}\textbf{表示}\\
18pt & \{{\textbackslash}fs\{18\} {\textbackslash}emblemdebian\} & \hspace{5mm}{\fs{18} \emblemdebian}\\
40pt & \{{\textbackslash}fs\{40\} {\textbackslash}emblemdebian\} & {\fs{40} \emblemdebian}\\
\end{tabular}
}
\end{table}
\end{itemize}

\newpage

\section{アイコンライブラリの使用:付録}
使用例:\href{https://fonts.google.com/icons}{Material Design Icons}\\
\hspace{4mm}詳細参照:\href{https://github.com/ru-museum/material-design-icons-latex}{material-design-icons-latex} \bsgithub

\begin{itemize}
  \item 他の公開されているオープンソースのアイコンライブラリでも同様の手順で利用出来ます。
\end{itemize}

\subsection{ICON ファイル(.svg)のダウンロード。}
\href{https://fonts.google.com/icons}{Material Design Icons}から使用するアイコンの SVG ファイルをダウンロードし所定のフォルダに保存します(例:icons)。

\begin{itemize}
  \item ファイル名は簡便なものに変更して置きます。\\
\hspace{4mm}例: thumb\_up\_black\_24dp.svg → thumb\_up.svg 
\end{itemize}

\subsection{パッケージファイルの編集。}
\textbf{materialicons.sty}\\  
{\fs{8}
\hspace{4mm}\% フォントサイズ:変更可能\vspace{1mm}\\
\hspace{4mm}\textbackslash def\textbackslash @icon\{\textbackslash includesvg[width=\textbf{1em}]\}\vspace{2mm}\\
\hspace{4mm}\% アイコンファイル(.svg)フォルダ:変更可能\vspace{1mm}\\
\hspace{4mm}\textbackslash def\textbackslash @dir\{\textbf{./icons}\}\vspace{2mm}\\
\hspace{4mm}\% 以下にファイル名に従い追加します\vspace{1mm}\\
\hspace{4mm}\textbackslash newcommand\{\textcolor{code}{\textbf{\textbackslash mdBuild}}\}\{\textbackslash@icon\{\textbackslash@dir/\textcolor{code}{\textbf{build}}.svg\}\}\\
\hspace{4mm}\textbackslash newcommand\{\textcolor{code}{\textbf{\textbackslash bsgithub}}\}\{\textbackslash@icon\{\textbackslash@dir/\textcolor{code}{\textbf{github}}.svg\}\} \% Bootstrap Icons

}

\subsection{TEX 内での表示。}
\% パッケージの読込\\
\hspace{4mm}\textbackslash usepackage\{materiaicons\}\\
\hspace{4mm}\textbackslash usepackage\{svg\} \% 必須\\

\% フォントサイズのマクロを定義\\
\hspace{4mm}\textbackslash def\textbackslash fs\#1\{\textbackslash fontsize\{\#1pt\}\{14pt\}\textbackslash selectfont\}
  
\begin{table}[H]
% \centering
\caption{アイコンの表示}
\begin{tabular}{lll}
アイコンライブラリ & デフォルト & サイズ指定\\
\hline\vspace{-4mm}\\
Material Design Icons & \textbackslash mdBuild \mdBuild & \{\textbackslash fs\{20\} \textbackslash mdBuild\}  {\fs{20} \mdBuild}\\
Bootstrap Icons & \textbackslash bsgithub \bsgithub & \{\textbackslash fs\{20\} \textbackslash bsgithub\} {\fs{20} \bsgithub}\\

\end{tabular}
\end{table}

\newpage

\section{gedit-latex ICONS 一覧}

\hspace{4mm} フォルダ:/usr/share/gedit/plugins/latex/icons/ (Debian)

\begin{itemize}
  \item command 名は基本的に prefix として \textbf{gl} を付与したものとなります。\vspace{-2mm}
  \item \_(アンダーバー)、-(ハイフン) は削除されます。
\end{itemize}

\begin{table}[H]
\fs{14pt}
\begin{tabular}{ll}
 ├── {\textbackslash}glabort \hspace{31mm} &\glabort\\
 ├── {\textbackslash}glbb & \glbb\\
 ├── {\textbackslash}glbf & \glbf\\
 ├── {\textbackslash}glbib & \glbib\\
 ├── {\textbackslash}glbraces & \glbraces\\
 ├── {\textbackslash}glbrackets & \glbrackets\\
 ├── {\textbackslash}glbuildimage & \glbuildimage\\
 ├── {\textbackslash}glcal & \glcal\\
 ├── {\textbackslash}glcalendar & \glcalendar\\
 ├── {\textbackslash}glchapter & \glchapter\\
 ├── {\textbackslash}glcloseenv & \glcloseenv\\
 ├── {\textbackslash}gldescription & \gldescription\\
 ├── {\textbackslash}gldisplaymath & \gldisplaymath\\
 ├── {\textbackslash}gldocument & \gldocument\\
 ├── {\textbackslash}gldocuments & \gldocuments\\
 ├── {\textbackslash}gldvbars & \gldvbars\\
 ├── {\textbackslash}glenumerate & \glenumerate\\
 ├── {\textbackslash}gleqnarray & \gleqnarray\\
 \end{tabular}
\end{table}

\newpage
 
\begin{table}[H]
\fs{14pt}
\begin{tabular}{ll}
 ├── {\textbackslash}glequation \hspace{24mm} & \glequation\\
 ├── {\textbackslash}glerror & \glerror\\
 ├── {\textbackslash}glfield & \glfield\\
 ├── {\textbackslash}glfrak & \glfrak\\
 ├── {\textbackslash}glgraphics & \glgraphics\\
 ├── {\textbackslash}glichoice & \glichoice\\
 ├── {\textbackslash}glicommand & \glicommand\\
 ├── {\textbackslash}glit & \glit\\
 ├── {\textbackslash}glitemize & \glitemize\\
 ├── {\textbackslash}gljustifycenter & \gljustifycenter\\
 ├── {\textbackslash}gljustifyleft & \gljustifyleft\\
 ├── {\textbackslash}gljustifyright & \gljustifyright\\
 ├── {\textbackslash}gllabel & \gllabel\\
 ├── {\textbackslash}gllisting & \gllisting\\
 ├── {\textbackslash}glmath & \glmath\\
 ├── {\textbackslash}glokay & \glokay\\
 ├── {\textbackslash}gloutline & \gloutline\\
 ├── {\textbackslash}glparagraph & \glparagraph\\
 ├── {\textbackslash}glparantheses & \glparantheses\\
 ├── {\textbackslash}glpart & \glpart\\
 ├── {\textbackslash}glrm & \glrm\\
 ├── {\textbackslash}glrun & \glrun\\
 ├── {\textbackslash}glsc & \glsc\\
 \end{tabular}
\end{table}  

\newpage
 
\begin{table}[H]
\vspace{-30mm}
\fs{14pt}
\begin{tabular}{ll}
 ├── {\textbackslash}glsection & \glsection\\
 ├── {\textbackslash}glsf & \glsf\\
 ├── {\textbackslash}glsnippet & \glsnippet\\
 ├── {\textbackslash}glsubsection & \glsubsection\\
 ├── {\textbackslash}glsubsubsection & \glsubsubsection\\
 ├── {\textbackslash}gltable & \gltable\\
 ├── {\textbackslash}gltask & \gltask\\
 ├── {\textbackslash}gltreechapter & \gltreechapter\\
 ├── {\textbackslash}gltreeinclude & \gltreeinclude\\
 ├── {\textbackslash}gltreeincludegraphics & \gltreeincludegraphics\\
 ├── {\textbackslash}gltreeparagraph & \gltreeparagraph\\
 ├── {\textbackslash}gltreepart & \gltreepart\\
 ├── {\textbackslash}gltreesection & \gltreesection\\
 ├── {\textbackslash}gltreesubsection & \gltreesubsection\\
 ├── {\textbackslash}gltreesubsubsection & \gltreesubsubsection\\
 ├── {\textbackslash}gltreetable & \gltreetable\\
 ├── {\textbackslash}gltt \hspace{37mm} & \gltt\\
 ├── {\textbackslash}glunderline & \glunderline\\
 ├── {\textbackslash}gluneqnarray & \gluneqnarray\\
 ├── {\textbackslash}glusers & \glusers\\
 ├── {\textbackslash}glvbars & \glvbars\\
 ├── {\textbackslash}glwarning & \glwarning\\
├── \textbf{ACCENTS} & \\
 ├── {\textbackslash}glacute \hspace{30mm} & \glacute\\
 ├── {\textbackslash}glbar & \glbar\\
 ├── {\textbackslash}glbreve & \glbreve\\
 ├── {\textbackslash}glcheck & \glcheck\\
 ├── {\textbackslash}glddot & \glddot\\
 \end{tabular}
\end{table}


\begin{table}[H]
\fs{14pt}
\begin{tabular}{ll}
 ├── {\textbackslash}gldot \hspace{36mm} & \gldot\\
 ├── {\textbackslash}glgrave & \glgrave\\
 ├── {\textbackslash}glhat & \glhat\\
 ├── {\textbackslash}gltilde & \gltilde\\
 ├── {\textbackslash}glvec & \glvec\\
├── \textbf{ARROWS} & \\
 ├── {\textbackslash}glLeftarrow & \glLeftarrow\\
 ├── {\textbackslash}glLeftrightarrow & \glLeftrightarrow\\
 ├── {\textbackslash}glLongleftarrow & \glLongleftarrow\\
 ├── {\textbackslash}glLongleftrightarrow & \glLongleftrightarrow\\
 ├── {\textbackslash}glLongrightarrow & \glLongrightarrow\\
 ├── {\textbackslash}glRightarrow & \glRightarrow\\
 ├── {\textbackslash}glhookleftarrow & \glhookleftarrow\\
 ├── {\textbackslash}glhookrightarrow & \glhookrightarrow\\
 ├── {\textbackslash}glleadsto & \glleadsto\\
 ├── {\textbackslash}glleftarrow & \glleftarrow\\
 ├── {\textbackslash}glleftharpoondown & \glleftharpoondown\\
 ├── {\textbackslash}glleftharpoonup & \glleftharpoonup\\
 ├── {\textbackslash}glleftrightarrow & \glleftrightarrow\\
 ├── {\textbackslash}glleftrightarrows & \glleftrightarrows\\
 ├── {\textbackslash}glleftrightharpoons & \glleftrightharpoons\\
 ├── {\textbackslash}gllongleftarrow & \gllongleftarrow\\
 ├── {\textbackslash}gllongleftrightarrow & \gllongleftrightarrow\\
 ├── {\textbackslash}gllongmapsto \hspace{15mm} & \gllongmapsto\\
 \end{tabular}
\end{table}

\newpage
 
\begin{table}[H]
\fs{14pt}
\begin{tabular}{ll}
 ├── {\textbackslash}gllongrightarrow \hspace{12mm} & \gllongrightarrow\\
 ├── {\textbackslash}glmapsto & \glmapsto\\
 ├── {\textbackslash}glrightarrow & \glrightarrow\\
 ├── {\textbackslash}glrightharpoondown & \glrightharpoondown\\
 ├── {\textbackslash}glrightharpoonup & \glrightharpoonup\\
 ├── {\textbackslash}glrightleftarrows & \glrightleftarrows\\
 ├── {\textbackslash}glrightleftharpoons & \glrightleftharpoons\\
├── \textbf{GREEK} & \\
 ├── {\textbackslash}glDelta & \glDelta\\
 ├── {\textbackslash}glGamma & \glGamma\\
 ├── {\textbackslash}glLambda & \glLambda\\
 ├── {\textbackslash}glOmega & \glOmega\\
 ├── {\textbackslash}glPhi & \glPhi\\
 ├── {\textbackslash}glPi & \glPi\\
 ├── {\textbackslash}glPsi & \glPsi\\
 ├── {\textbackslash}glSigma & \glSigma\\
 ├── {\textbackslash}glTheta & \glTheta\\
 ├── {\textbackslash}glUpsilon & \glUpsilon\\
 ├── {\textbackslash}glXi & \glXi\\
 ├── {\textbackslash}glalpha & \glalpha\\
 ├── {\textbackslash}glbeta & \glbeta\\
 ├── {\textbackslash}glchi & \glchi\\
 ├── {\textbackslash}gldelta & \gldelta\\
 \end{tabular}
\end{table}

\newpage
 
\begin{table}[H]
\fs{14pt}
\begin{tabular}{ll}
 ├── {\textbackslash}glepsilon \hspace{28mm} & \glepsilon\\
 ├── {\textbackslash}gleta & \gleta\\
 ├── {\textbackslash}glgamma & \glgamma\\
 ├── {\textbackslash}gliota & \gliota\\
 ├── {\textbackslash}glkappa & \glkappa\\
 ├── {\textbackslash}gllambda & \gllambda\\
 ├── {\textbackslash}glmu \hspace{32mm} & \glmu\\
 ├── {\textbackslash}glnu & \glnu\\
 ├── {\textbackslash}glomega & \glomega\\
 ├── {\textbackslash}glphi & \glphi\\
 ├── {\textbackslash}glpi & \glpi\\
 ├── {\textbackslash}glpsi & \glpsi\\
 ├── {\textbackslash}glrho & \glrho\\
 ├── {\textbackslash}glsigma & \glsigma\\
 ├── {\textbackslash}gltau & \gltau\\
 ├── {\textbackslash}gltheta & \gltheta\\
 ├── {\textbackslash}glupsilon & \glupsilon\\
 ├── {\textbackslash}glvarepsilon & \glvarepsilon\\
 ├── {\textbackslash}glvarphi & \glvarphi\\
 ├── {\textbackslash}glvarpi & \glvarpi\\
 ├── {\textbackslash}glvarrho & \glvarrho\\
 ├── {\textbackslash}glvarsigma & \glvarsigma\\
 ├── {\textbackslash}glvartheta & \glvartheta\\
 \end{tabular}
\end{table}

\newpage
 
\begin{table}[H]
\fs{14pt}
\begin{tabular}{ll}
 ├── {\textbackslash}glxi \hspace{39mm} & \glxi\\
 ├── {\textbackslash}glzeta & \glzeta\\
├── \textbf{LIMITERS} & \\
 ├── {\textbackslash}glVert & \glVert\\
 ├── {\textbackslash}gllangle & \gllangle\\
 ├── {\textbackslash}gllbrace & \gllbrace\\
 ├── {\textbackslash}gllceil & \gllceil\\
 ├── {\textbackslash}gllfloor & \gllfloor\\
 ├── {\textbackslash}glrangle & \glrangle\\
 ├── {\textbackslash}glrbrace & \glrbrace\\
 ├── {\textbackslash}glrceil & \glrceil\\
 ├── {\textbackslash}glrfloor & \glrfloor\\
 ├── {\textbackslash}glvert & \glvert\\
├── \textbf{OPERATORS} & \\
 ├── {\textbackslash}glapprox & \glapprox\\
 ├── {\textbackslash}glast & \glast\\
 ├── {\textbackslash}glbullet & \glbullet\\
 ├── {\textbackslash}glcap & \glcap\\
 ├── {\textbackslash}glcdot & \glcdot\\
 ├── {\textbackslash}glcirc & \glcirc\\
 ├── {\textbackslash}glcong & \glcong\\
 ├── {\textbackslash}glcup & \glcup\\
 ├── {\textbackslash}gldiv & \gldiv\\
 ├── {\textbackslash}glequiv & \glequiv\\
 \end{tabular}
\end{table}

\newpage
 
\begin{table}[H]
\fs{14pt}
\begin{tabular}{ll}
 ├── {\textbackslash}gleth \hspace{36mm} & \gleth\\
 ├── {\textbackslash}glexists & \glexists\\
 ├── {\textbackslash}glforall & \glforall\\
 ├── {\textbackslash}glgeq & \glgeq\\
 ├── {\textbackslash}glgg & \glgg\\
 ├── {\textbackslash}glin & \glin\\
 ├── {\textbackslash}glland & \glland\\
 ├── {\textbackslash}glleq & \glleq\\
 ├── {\textbackslash}glll & \glll\\
 ├── {\textbackslash}gllor & \gllor\\
 ├── {\textbackslash}glmid & \glmid\\
 ├── {\textbackslash}glnabla & \glnabla\\
 ├── {\textbackslash}glneg & \glneg\\
 ├── {\textbackslash}glneq & \glneq\\
 ├── {\textbackslash}glnexists & \glnexists\\
 ├── {\textbackslash}glni & \glni\\
 ├── {\textbackslash}gloplus & \gloplus\\
 ├── {\textbackslash}glparallel & \glparallel\\
 ├── {\textbackslash}glpartial & \glpartial\\
 ├── {\textbackslash}glperp & \glperp\\
 ├── {\textbackslash}glpm & \glpm\\
 ├── {\textbackslash}glpropto & \glpropto\\
 \end{tabular}
\end{table}

\newpage
 
\begin{table}[H]
\fs{14pt}
\begin{tabular}{ll}
 ├── {\textbackslash}glsetminus \hspace{24mm} & \glsetminus\\
 ├── {\textbackslash}glsim & \glsim\\
 ├── {\textbackslash}glsimeq & \glsimeq\\
 ├── {\textbackslash}glstar & \glstar\\
 ├── {\textbackslash}glsubset & \glsubset\\
 ├── {\textbackslash}glsubseteq & \glsubseteq\\
 ├── {\textbackslash}glsupset & \glsupset\\
 ├── {\textbackslash}glsupseteq & \glsupseteq\\
 ├── {\textbackslash}gltimes & \gltimes\\
 ├── {\textbackslash}gluplus & \gluplus\\
 ├── {\textbackslash}glvarnothing & \glvarnothing\\
├── \textbf{SPECIAL} & \\
 ├── {\textbackslash}glIm & \glIm\\
 ├── {\textbackslash}glP & \glP\\
 ├── {\textbackslash}glRe & \glRe\\
 ├── {\textbackslash}glS & \glS\\
 ├── {\textbackslash}glaleph & \glaleph\\
 ├── {\textbackslash}glangle & \glangle\\
 ├── {\textbackslash}glcdots & \glcdots\\
 ├── {\textbackslash}glcomplement & \glcomplement\\
 ├── {\textbackslash}gldagger & \gldagger\\
 \end{tabular}
\end{table}
 
\newpage
 
\begin{table}[H]
\fs{14pt}
\begin{tabular}{ll}
 ├── {\textbackslash}glddots \hspace{31mm} & \glddots\\
 ├── {\textbackslash}glell & \glell\\ 
 ├── {\textbackslash}glhbar & \glhbar\\
 ├── {\textbackslash}gliint & \gliint\\
 ├── {\textbackslash}glinfty & \glinfty\\
 ├── {\textbackslash}glint & \glint\\
 ├── {\textbackslash}glldots & \glldots\\
 ├── {\textbackslash}gloint & \gloint\\
 ├── {\textbackslash}glprime & \glprime\\
 ├── {\textbackslash}glprod & \glprod\\
 ├── {\textbackslash}glsum & \glsum\\
 ├── {\textbackslash}gltextcopyright & \gltextcopyright\\
 ├── {\textbackslash}gltextregistered & \gltextregistered\\
 ├── {\textbackslash}gltexttrademark & \gltexttrademark\\
 ├── {\textbackslash}glvdots & \glvdots\\
 ├── {\textbackslash}glwp & \glwp\\

 \end{tabular}
\end{table}

\end{document}
