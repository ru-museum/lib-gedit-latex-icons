\documentclass[a5j,10pt]{ltjarticle}
\usepackage{fontspec}

% LINK
\usepackage{url}
\usepackage{hyperref}

% 色の使用
\usepackage{xcolor}
\definecolor{code}{HTML}{800000}
\definecolor{mylinkcolor}{RGB}{3, 112, 145} % 色定義
\definecolor{lightblu}{HTML}{004c8c} 
\definecolor{orangish}{wave}{620}
\colorlet{ochre}{blue!30!yellow!70!}

\hypersetup{
    colorlinks=true,
    citecolor=blue,
    linkcolor=lightblu,
    urlcolor=mylinkcolor, % 定義された色 MidnightBlue
}

% 表の使用
\usepackage{float}

% ICONS Lib 読込
\usepackage{lib-gedit-latex-icons} 
\usepackage{graphicx} 
\usepackage{svg} 

\usepackage{datetime2}
\usepackage{setspace}

% FONT-SIZE 定義
\def\fs#1{\fontsize{#1pt}{14pt}\selectfont}

% サンプル定義
% DESKTOP-ICONS
\newcommand{\emblemdebian}{\includegraphics[width=1em]{/usr/share/icons/desktop-base/64x64/emblems/emblem-debian.png}}
% MATERIAL-ICONS
\newcommand{\mdBuild}{\includesvg[width=1em]{./icons/build.svg}}
% BOOTSTRAP-ICONS
\newcommand{\bsgithub}{\includesvg[width=1em]{./icons/github.svg}}

% TITLEPAGE
\title{{\fs{24.88}GEDIT-LaTeX-Plugin \\ ICONS for \LaTeX{}}\\ \vspace{10mm}{\fs{12} \textbf{[ PNG 版 ]}}\vspace{80mm}}
\author{\href{https://github.com/ru-museum/}{ru\_museum \bsgithub}} 
\date{\today}

\begin{document}

\begin{titlepage}
  \maketitle
	\vspace{\fill}
\thispagestyle{empty}
\end{titlepage}

\newpage

\tableofcontents
\clearpage

\newpage

\section{"gedit-latex-plugin"}
\subsection{概要}

\begin{itemize}
  \item LINUX 用エディタ \textbf{gedit} には\LaTeX{}向けのプラグイン \textbf{gedit-latex-plugin} があり、添付の ICON が利用出来ます。\\ 
※ 参照: 「3 gedit-latex ICONS 一覧」
  
  \item ICON 収納 PATH(LINUX Debian の場合):\\
  \textbf{/usr/share/gedit/plugins/latex/icons/} 

\end{itemize}

\subsection{インストールと設定}

\begin{itemize}
  \item インストール\\ 
        \$ apt-get install gedit-latex-plugin (Debian の場合) \\
        {\fs{8} ※ PDF viewer(evince 等)、inkscape(SVG) が必要な場合があります。} 
\vspace{-2mm}
  \item プラグインの設定で SyncTeX をオンにして置きます(作業用)。\\ 
{\fs{8}SyncTeX:gedit と evince(PDF viewer) で LaTeX と PDF を同期します。}
\end{itemize}

\subsection{使用方法}

\subsubsection{ライブラリの読込}
{\textbackslash}usepackage\{lib-gedit-latex-icons\} \% ICON 定義ライブラリ \\ 
 {\textbackslash}usepackage\{graphicx\}\hspace{18mm}\% PNG 用\\ 
 {\textbackslash}usepackage\{svg\}\hspace{26mm}\% SVG 使用時(inkscape 必須)\\ 

\% 色を使用する場合\\
 {\textbackslash}usepackage\{xcolor\}\\
 {\textbackslash}definecolor\{code\}\{HTML\}\{800000\}\\
 {\textbackslash}definecolor\{mylinkcolor\}\{RGB\}\{3, 112, 145\}

\subsubsection{定義}
// フォントサイズのマクロ定義 \\
 {\textbackslash}def{\textbackslash}fs\#1\{{\textbackslash}fontsize\{\#1pt\}\{14pt\}{\textbackslash}selectfont\}\\

// ICON のマクロ定義\\
 {\textbackslash}newcommand\{{\textbackslash}\textbf{gl}closeenv\}\{{\textbackslash}@icon\{{\textbackslash}@dir/close\_env.png\}\}\\
\begin{itemize}\vspace{-6mm}
  \item システムとの衝突を避ける為、prefix として \textbf{gl} を付与しています。
\vspace{-3mm}
 \item command 名はファイル名に準じ \_(アンダーバー)、-(ハイフン) 等は使用出来ませんので削除しています。
\end{itemize}


\subsubsection{表示}

\begin{itemize}

  \item 各々定義した command 名で表示させます。\\
  ※ ファイルは基本 PNG 形式ですので過度の拡大は像が粗くなります。 

\begin{table}[H]
\caption{表示例} %キャプション
\vspace{4mm}
{\hspace{10mm}
\begin{tabular}{lll}
\textbf{サイズ} & \textbf{記述} & \hspace{4mm}\textbf{表示}\\
無指定 & {\textbackslash}glabort & \hspace{8mm}\glabort\\
{\fs{18}18pt} & \{{\textbackslash}fs\{18\} {\textbackslash}glabort\} & \hspace{5mm}{\fs{18} \glabort}\\
{\fs{40}40pt} & \{{\textbackslash}fs\{40\} {\textbackslash}glabort\} & {\fs{40} \glabort}\\
\end{tabular}
}
\end{table}
  
\vspace{-3mm}
  \item \LaTeX{} 付属のシステムアイコンは、\textbf{text-} で使うことが出来ます。\\
{\textbackslash}textparagraph: \textparagraph \\
{\textbackslash}textcopyright : \textcopyright
  \item その他、OS のデスクトップ環境における各種テーマのアイコンセットも同様に使うことが出来ます。\\ 
指定PATH:/usr/share/icons/... (Debian の場合)\\

// 定義例\\
{\textbackslash}newcommand\{{\textbackslash}emblemdebian\}\{/usr/share/icons/desktop-base/\\ 64x64/emblems/emblem-debian.png\}\} \\

\vspace{-6mm}
\begin{table}[H]
\caption{表示例}
\vspace{4mm}
{\hspace{10mm}
\begin{tabular}{lll}
\textbf{サイズ} & \textbf{記述} & \hspace{4mm}\textbf{表示}\\
18pt & \{{\textbackslash}fs\{18\} {\textbackslash}emblemdebian\} & \hspace{5mm}{\fs{18} \emblemdebian}\\
40pt & \{{\textbackslash}fs\{40\} {\textbackslash}emblemdebian\} & {\fs{40} \emblemdebian}\\
\end{tabular}
}
\end{table}
\end{itemize}

\newpage

\section{アイコンライブラリの使用:付録}
使用例:\href{https://fonts.google.com/icons}{Material Design Icons}\\
\hspace{4mm}詳細参照:\href{https://github.com/ru-museum/material-design-icons-latex}{material-design-icons-latex} \bsgithub

\begin{itemize}
  \item 他の公開されているオープンソースのアイコンライブラリでも同様の手順で利用出来ます。
\end{itemize}

\subsection{ICON ファイル(.svg)のダウンロード。}
\href{https://fonts.google.com/icons}{Material Design Icons}から使用するアイコンの SVG ファイルをダウンロードし所定のフォルダに保存します(例:icons)。

\begin{itemize}
  \item ファイル名は簡便なものに変更して置きます。\\
\hspace{4mm}例: thumb\_up\_black\_24dp.svg → thumb\_up.svg 
\end{itemize}

\subsection{パッケージファイルの編集。}
\textbf{materialicons.sty}\\  
{\fs{8}
\hspace{4mm}\% フォントサイズ:変更可能\vspace{1mm}\\
\hspace{4mm}\textbackslash def\textbackslash @icon\{\textbackslash includesvg[width=\textbf{1em}]\}\vspace{2mm}\\
\hspace{4mm}\% アイコンファイル(.svg)フォルダ:変更可能\vspace{1mm}\\
\hspace{4mm}\textbackslash def\textbackslash @dir\{\textbf{./icons}\}\vspace{2mm}\\
\hspace{4mm}\% 以下にファイル名に従い追加します\vspace{1mm}\\
\hspace{4mm}\textbackslash newcommand\{\textcolor{code}{\textbf{\textbackslash mdBuild}}\}\{\textbackslash@icon\{\textbackslash@dir/\textcolor{code}{\textbf{build}}.svg\}\}\\
\hspace{4mm}\textbackslash newcommand\{\textcolor{code}{\textbf{\textbackslash bsgithub}}\}\{\textbackslash@icon\{\textbackslash@dir/\textcolor{code}{\textbf{github}}.svg\}\} \% Bootstrap Icons

}

\subsection{TEX 内での表示。}
\% パッケージの読込\\
\hspace{4mm}\textbackslash usepackage\{materiaicons\}\\
\hspace{4mm}\textbackslash usepackage\{svg\} \% 必須\\

\% フォントサイズのマクロを定義\\
\hspace{4mm}\textbackslash def\textbackslash fs\#1\{\textbackslash fontsize\{\#1pt\}\{14pt\}\textbackslash selectfont\}
  
\begin{table}[H]
% \centering
\caption{アイコンの表示}
\begin{tabular}{lll}
アイコンライブラリ & デフォルト & サイズ指定\\
\hline\vspace{-4mm}\\
Material Design Icons & \textbackslash mdBuild \mdBuild & \{\textbackslash fs\{20\} \textbackslash mdBuild\}  {\fs{20} \mdBuild}\\
Bootstrap Icons & \textbackslash bsgithub \bsgithub & \{\textbackslash fs\{20\} \textbackslash bsgithub\} {\fs{20} \bsgithub}\\

\end{tabular}
\end{table}

\newpage

\section{gedit-latex ICONS 一覧}

\hspace{4mm} フォルダ:/usr/share/gedit/plugins/latex/icons/ (Debian)

\begin{itemize}
  \item command 名は基本的に prefix として \textbf{gl} を付与したものとなります。\vspace{-2mm}
  \item \_(アンダーバー)、-(ハイフン) は削除されます。
\end{itemize}

\begin{table}[H]
\fs{14pt}
\begin{tabular}{ll}
 ├── abort.png \hspace{31mm} &\glabort\\
 ├── bb.png & \glbb\\
 ├── bf.png & \glbf\\
 ├── bib.png & \glbib\\
 ├── braces.png & \glbraces\\
 ├── brackets.png & \glbrackets\\
 ├── build\-image.png & \glbuildimage\\
 ├── cal.png & \glcal\\
 ├── calendar.png & \glcalendar\\
 ├── chapter.png & \glchapter\\
 ├── close\_env.png & \glcloseenv\\
 ├── description.png & \gldescription\\
 ├── displaymath.png & \gldisplaymath\\
 ├── document.png & \gldocument\\
 ├── documents.png & \gldocuments\\
 ├── dvbars.png & \gldvbars\\
 ├── enumerate.png & \glenumerate\\
 ├── eqnarray.png & \gleqnarray\\
 \end{tabular}
\end{table}

\newpage
 
\begin{table}[H]
\fs{14pt}
\begin{tabular}{ll}
 ├── equation.png \hspace{24mm} & \glequation\\
 ├── error.png & \glerror\\
 ├── field.png & \glfield\\
 ├── frak.png & \glfrak\\
 ├── graphics.png & \glgraphics\\
 ├── i\_choice.png & \glichoice\\
 ├── i\_command.png & \glicommand\\
 ├── it.png & \glit\\
 ├── itemize.png & \glitemize\\
 ├── justify\-center.png & \gljustifycenter\\
 ├── justify\-left.png & \gljustifyleft\\
 ├── justify\-right.png & \gljustifyright\\
 ├── label.png & \gllabel\\
 ├── listing.png & \gllisting\\
 ├── math.png & \glmath\\
 ├── okay.png & \glokay\\
 ├── outline.png & \gloutline\\
 ├── paragraph.png & \glparagraph\\
 ├── parantheses.png & \glparantheses\\
 ├── part.png & \glpart\\
 ├── rm.png & \glrm\\
 ├── run.png & \glrun\\
 ├── sc.png & \glsc\\
 \end{tabular}
\end{table}  

\newpage
 
\begin{table}[H]
\vspace{-30mm}
\fs{14pt}
\begin{tabular}{ll}
 ├── section.png & \glsection\\
 ├── sf.png & \glsf\\
 ├── snippet.png & \glsnippet\\
 ├── subsection.png & \glsubsection\\
 ├── subsubsection.png & \glsubsubsection\\
 ├── table.png & \gltable\\
 ├── task.png & \gltask\\
 ├── tree\_chapter.png & \gltreechapter\\
 ├── tree\_include.png & \gltreeinclude\\
 ├── tree\_includegraphics.png & \gltreeincludegraphics\\
 ├── tree\_paragraph.png & \gltreeparagraph\\
 ├── tree\_part.png & \gltreepart\\
 ├── tree\_section.png & \gltreesection\\
 ├── tree\_subsection.png & \gltreesubsection\\
 ├── tree\_subsubsection.png & \gltreesubsubsection\\
 ├── tree\_table.png & \gltreetable\\
 ├── tt.png \hspace{37mm} & \gltt\\
 ├── underline.png & \glunderline\\
 ├── uneqnarray.png & \gluneqnarray\\
 ├── users.png & \glusers\\
 ├── vbars.png & \glvbars\\
 ├── warning.png & \glwarning\\
├── \textbf{ACCENTS} & \\
 ├── acute.png \hspace{30mm} & \glacute\\
 ├── bar.png & \glbar\\
 ├── breve.png & \glbreve\\
 ├── check.png & \glcheck\\
 ├── ddot.png & \glddot\\
 \end{tabular}
\end{table}


\begin{table}[H]
\fs{14pt}
\begin{tabular}{ll}
 ├── dot.png \hspace{36mm} & \gldot\\
 ├── grave.png & \glgrave\\
 ├── hat.png & \glhat\\
 ├── tilde.png & \gltilde\\
 ├── vec.png & \glvec\\
├── \textbf{ARROWS} & \\
 ├── Leftarrow.png & \glLeftarrow\\
 ├── Leftrightarrow.png & \glLeftrightarrow\\
 ├── Longleftarrow.png & \glLongleftarrow\\
 ├── Longleftrightarrow.png & \glLongleftrightarrow\\
 ├── Longrightarrow.png & \glLongrightarrow\\
 ├── Rightarrow.png & \glRightarrow\\
 ├── hookleftarrow.png & \glhookleftarrow\\
 ├── hookrightarrow.png & \glhookrightarrow\\
 ├── leadsto.png & \glleadsto\\
 ├── leftarrow.png & \glleftarrow\\
 ├── leftharpoondown.png & \glleftharpoondown\\
 ├── leftharpoonup.png & \glleftharpoonup\\
 ├── leftrightarrow.png & \glleftrightarrow\\
 ├── leftrightarrows.png & \glleftrightarrows\\
 ├── leftrightharpoons.png & \glleftrightharpoons\\
 ├── longleftarrow.png & \gllongleftarrow\\
 ├── longleftrightarrow.png & \gllongleftrightarrow\\
 ├── longmapsto.png \hspace{15mm} & \gllongmapsto\\
 \end{tabular}
\end{table}

\newpage
 
\begin{table}[H]
\fs{14pt}
\begin{tabular}{ll}
 ├── longrightarrow.png \hspace{12mm} & \gllongrightarrow\\
 ├── mapsto.png & \glmapsto\\
 ├── rightarrow.png & \glrightarrow\\
 ├── rightharpoondown.png & \glrightharpoondown\\
 ├── rightharpoonup.png & \glrightharpoonup\\
 ├── rightleftarrows.png & \glrightleftarrows\\
 ├── rightleftharpoons.png & \glrightleftharpoons\\
├── \textbf{GREEK} & \\
 ├── Delta.png & \glDelta\\
 ├── Gamma.png & \glGamma\\
 ├── Lambda.png & \glLambda\\
 ├── Omega.png & \glOmega\\
 ├── Phi.png & \glPhi\\
 ├── Pi.png & \glPi\\
 ├── Psi.png & \glPsi\\
 ├── Sigma.png & \glSigma\\
 ├── Theta.png & \glTheta\\
 ├── Upsilon.png & \glUpsilon\\
 ├── Xi.png & \glXi\\
 ├── alpha.png & \glalpha\\
 ├── beta.png & \glbeta\\
 ├── chi.png & \glchi\\
 ├── delta.png & \gldelta\\
 \end{tabular}
\end{table}

\newpage
 
\begin{table}[H]
\fs{14pt}
\begin{tabular}{ll}
 ├── epsilon.png \hspace{28mm} & \glepsilon\\
 ├── eta.png & \gleta\\
 ├── gamma.png & \glgamma\\
 ├── iota.png & \gliota\\
 ├── kappa.png & \glkappa\\
 ├── lambda.png & \gllambda\\
 ├── mu.png \hspace{32mm} & \glmu\\
 ├── nu.png & \glnu\\
 ├── omega.png & \glomega\\
 ├── phi.png & \glphi\\
 ├── pi.png & \glpi\\
 ├── psi.png & \glpsi\\
 ├── rho.png & \glrho\\
 ├── sigma.png & \glsigma\\
 ├── tau.png & \gltau\\
 ├── theta.png & \gltheta\\
 ├── upsilon.png & \glupsilon\\
 ├── varepsilon.png & \glvarepsilon\\
 ├── varphi.png & \glvarphi\\
 ├── varpi.png & \glvarpi\\
 ├── varrho.png & \glvarrho\\
 ├── varsigma.png & \glvarsigma\\
 ├── vartheta.png & \glvartheta\\
 \end{tabular}
\end{table}

\newpage
 
\begin{table}[H]
\fs{14pt}
\begin{tabular}{ll}
 ├── xi.png \hspace{39mm} & \glxi\\
 ├── zeta.png & \glzeta\\
├── \textbf{LIMITERS} & \\
 ├── Vert.png & \glVert\\
 ├── langle.png & \gllangle\\
 ├── lbrace.png & \gllbrace\\
 ├── lceil.png & \gllceil\\
 ├── lfloor.png & \gllfloor\\
 ├── rangle.png & \glrangle\\
 ├── rbrace.png & \glrbrace\\
 ├── rceil.png & \glrceil\\
 ├── rfloor.png & \glrfloor\\
 ├── vert.png & \glvert\\
├── \textbf{OPERATORS} & \\
 ├── approx.png & \glapprox\\
 ├── ast.png & \glast\\
 ├── bullet.png & \glbullet\\
 ├── cap.png & \glcap\\
 ├── cdot.png & \glcdot\\
 ├── circ.png & \glcirc\\
 ├── cong.png & \glcong\\
 ├── cup.png & \glcup\\
 ├── div.png & \gldiv\\
 ├── equiv.png & \glequiv\\
 \end{tabular}
\end{table}

\newpage
 
\begin{table}[H]
\fs{14pt}
\begin{tabular}{ll}
 ├── eth.png \hspace{36mm} & \gleth\\
 ├── exists.png & \glexists\\
 ├── forall.png & \glforall\\
 ├── geq.png & \glgeq\\
 ├── gg.png & \glgg\\
 ├── in.png & \glin\\
 ├── land.png & \glland\\
 ├── leq.png & \glleq\\
 ├── ll.png & \glll\\
 ├── lor.png & \gllor\\
 ├── mid.png & \glmid\\
 ├── nabla.png & \glnabla\\
 ├── neg.png & \glneg\\
 ├── neq.png & \glneq\\
 ├── nexists.png & \glnexists\\
 ├── ni.png & \glni\\
 ├── oplus.png & \gloplus\\
 ├── parallel.png & \glparallel\\
 ├── partial.png & \glpartial\\
 ├── perp.png & \glperp\\
 ├── pm.png & \glpm\\
 ├── propto.png & \glpropto\\
 \end{tabular}
\end{table}

\newpage
 
\begin{table}[H]
\fs{14pt}
\begin{tabular}{ll}
 ├── setminus.png \hspace{24mm} & \glsetminus\\
 ├── sim.png & \glsim\\
 ├── simeq.png & \glsimeq\\
 ├── star.png & \glstar\\
 ├── subset.png & \glsubset\\
 ├── subseteq.png & \glsubseteq\\
 ├── supset.png & \glsupset\\
 ├── supseteq.png & \glsupseteq\\
 ├── times.png & \gltimes\\
 ├── uplus.png & \gluplus\\
 ├── varnothing.png & \glvarnothing\\
├── \textbf{SPECIAL} & \\
 ├── Im.png & \glIm\\
 ├── P.png & \glP\\
 ├── Re.png & \glRe\\
 ├── S.png & \glS\\
 ├── aleph.png & \glaleph\\
 ├── angle.png & \glangle\\
 ├── cdots.png & \glcdots\\
 ├── complement.png & \glcomplement\\
 ├── dagger.png & \gldagger\\
 \end{tabular}
\end{table}
 
\newpage
 
\begin{table}[H]
\fs{14pt}
\begin{tabular}{ll}
 ├── ddots.png \hspace{31mm} & \glddots\\
 ├── ell.png & \glell\\ 
 ├── hbar.png & \glhbar\\
 ├── iint.png & \gliint\\
 ├── infty.png & \glinfty\\
 ├── int.png & \glint\\
 ├── ldots.png & \glldots\\
 ├── oint.png & \gloint\\
 ├── prime.png & \glprime\\
 ├── prod.png & \glprod\\
 ├── sum.png & \glsum\\
 ├── textcopyright.png & \gltextcopyright\\
 ├── textregistered.png & \gltextregistered\\
 ├── texttrademark.png & \gltexttrademark\\
 ├── vdots.png & \glvdots\\
 ├── wp.png & \glwp\\

 \end{tabular}
\end{table}

\end{document}
